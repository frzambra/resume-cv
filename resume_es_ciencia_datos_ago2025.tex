
%-----------------------------------------------------------------------------------------------------------------------------------------------%
%	The MIT License (MIT)
%
%	Copyright (c) 2019 Jan Küster
%
%	Permission is hereby granted, free of charge, to any person obtaining a copy
%	of this software and associated documentation files (the "Software"), to deal
%	in the Software without restriction, including without limitation the rights
%	to use, copy, modify, merge, publish, distribute, sublicense, and/or sell
%	copies of the Software, and to permit persons to whom the Software is
%	furnished to do so, subject to the following conditions:
%	
%	THE SOFTWARE IS PROVIDED "AS IS", WITHOUT WARRANTY OF ANY KIND, EXPRESS OR
%	IMPLIED, INCLUDING BUT NOT LIMITED TO THE WARRANTIES OF MERCHANTABILITY,
%	FITNESS FOR A PARTICULAR PURPOSE AND NONINFRINGEMENT. IN NO EVENT SHALL THE
%	AUTHORS OR COPYRIGHT HOLDERS BE LIABLE FOR ANY CLAIM, DAMAGES OR OTHER
%	LIABILITY, WHETHER IN AN ACTION OF CONTRACT, TORT OR OTHERWISE, ARISING FROM,
%	OUT OF OR IN CONNECTION WITH THE SOFTWARE OR THE USE OR OTHER DEALINGS IN
%	THE SOFTWARE.
%	
%
%-----------------------------------------------------------------------------------------------------------------------------------------------%


%============================================================================%
%
%	DOCUMENT DEFINITION
%
%============================================================================%

\documentclass[10pt,A4,english]{article}	


%----------------------------------------------------------------------------------------
%	ENCODING
%----------------------------------------------------------------------------------------

% we use utf8 since we want to build from any machine
\usepackage[utf8]{inputenc}		
\usepackage[USenglish]{isodate}
\usepackage{fancyhdr}
% \usepackage[numbers]{natbib}

%----------------------------------------------------------------------------------------
%	LOGIC
%----------------------------------------------------------------------------------------

% provides \isempty test
\usepackage{xstring, xifthen}
\usepackage{enumitem}
\usepackage[english]{babel}
\usepackage{blindtext}
\usepackage{pdfpages}
\usepackage{changepage}
%----------------------------------------------------------------------------------------
%	FONT BASICS
%----------------------------------------------------------------------------------------

% some tex-live fonts - choose your own
\usepackage[default]{raleway}

% set font default
\renewcommand*\familydefault{\sfdefault} 	
\usepackage[T1]{fontenc}

% more font size definitions
\usepackage{moresize}

%----------------------------------------------------------------------------------------
%	FONT AWESOME ICONS
%---------------------------------------------------------------------------------------- 

% include the fontawesome icon set
\usepackage{fontawesome5}

% use to vertically center content
% credits to: http://tex.stackexchange.com/questions/7219/how-to-vertically-center-two-images-next-to-each-other
\newcommand{\vcenteredinclude}[1]{\begingroup
\setbox0=\hbox{\includegraphics{#1}}%
\parbox{\wd0}{\box0}\endgroup}
\newcommand{\tab}[1]{\hspace{.2\textwidth}\rlap{#1}}
% use to vertically center content
% credits to: http://tex.stackexchange.com/questions/7219/how-to-vertically-center-two-images-next-to-each-other
\newcommand*{\vcenteredhbox}[1]{\begingroup
\setbox0=\hbox{#1}\parbox{\wd0}{\box0}\endgroup}

% icon shortcut
\newcommand{\icon}[2] { 							
	\makebox(#2, #2){\textcolor{maincol}{\textcolor{maincol}{\faIcon{#1}}}}
}	


% icon with text shortcut
\newcommand{\icontext}[3]{ 						
	\vcenteredhbox{\icon{#1}{#2}}  \hspace{2pt}  \parbox{0.9\mpwidth}{\textcolor{black}{#3}}
}

% icon with website url
\newcommand{\iconhref}[4]{ 						
    \vcenteredhbox{\icon{#1}{#2}}  \hspace{2pt} \href{#4}{\textcolor{black}{#3}}
}

% icon with email link
\newcommand{\iconemail}[5]{ 						
    \vcenteredhbox{\icon{#1}{#2}{#5}}  \hspace{2pt} \href{mailto:#4}{\textcolor{#5}{#3}}
}

%----------------------------------------------------------------------------------------
%	PAGE LAYOUT  DEFINITIONS
%----------------------------------------------------------------------------------------

% page outer frames (debug-only)
% \usepackage{showframe}		

% we use paracol to display breakable two columns
\usepackage{paracol}
\usepackage{tikzpagenodes}
\usetikzlibrary{calc}
\usepackage{lmodern}
\usepackage{multicol}
\usepackage{lipsum}
\usepackage{atbegshi}
% define page styles using geometry
\usepackage[a4paper]{geometry}

% remove all possible margins
\geometry{top=1cm, bottom=1cm, left=1cm, right=1cm}

\usepackage{fancyhdr}
\pagestyle{empty}

% space between header and content
% \setlength{\headheight}{0pt}

% indentation is zero
\setlength{\parindent}{0mm}

%----------------------------------------------------------------------------------------
%	TABLE /ARRAY DEFINITIONS
%---------------------------------------------------------------------------------------- 

% extended aligning of tabular cells
\usepackage{array}

% custom column right-align with fixed width
% use like p{size} but via x{size}
\newcolumntype{x}[1]{%
>{\raggedleft\hspace{0pt}}p{#1}}%


%----------------------------------------------------------------------------------------
%	GRAPHICS DEFINITIONS
%---------------------------------------------------------------------------------------- 

%for header image
\usepackage{graphicx}

% use this for floating figures
% \usepackage{wrapfig}
% \usepackage{float}
% \floatstyle{boxed} 
% \restylefloat{figure}

%for drawing graphics		
\usepackage{tikz}			
\usepackage{ragged2e}	
\usetikzlibrary{shapes, backgrounds,mindmap, trees}

%----------------------------------------------------------------------------------------
%	Bibliography
%---------------------------------------------------------------------------------------- 

%----------------------------------------------------------------------------------------
%	Color DEFINITIONS
%---------------------------------------------------------------------------------------- 
\usepackage{transparent}
\usepackage{color}

% primary color
\definecolor{maincol}{HTML}{0A084D}

% accent color, secondary
% \definecolor{accentcol}{RGB}{ 250, 150, 10 }

% dark color
\definecolor{darkcol}{RGB}{ 70, 70, 70 }

% light color
\definecolor{lightcol}{RGB}{245,245,245}

\definecolor{accentcol}{HTML}{084D0B}



% Package for links, must be the last package used
\usepackage[hidelinks]{hyperref}

% returns minipage width minus two times \fboxsep
% to keep padding included in width calculations
% can also be used for other boxes / environments
\newcommand{\mpwidth}{\linewidth-\fboxsep-\fboxsep}
	


%============================================================================%
%
%	CV COMMANDS
%
%============================================================================%

%----------------------------------------------------------------------------------------
%	 CV LIST
%----------------------------------------------------------------------------------------

% renders a standard latex list but abstracts away the environment definition (begin/end)
\newcommand{\cvlist}[1] {
	\begin{itemize}{#1}\end{itemize}
}

%----------------------------------------------------------------------------------------
%	 CV TEXT
%----------------------------------------------------------------------------------------

% base class to wrap any text based stuff here. Renders like a paragraph.
% Allows complex commands to be passed, too.
% param 1: *any
\newcommand{\cvtext}[1] {
	\begin{tabular*}{1\mpwidth}{p{0.98\mpwidth}}
		\parbox{1\mpwidth}{#1}
	\end{tabular*}
}
\newcommand{\cvtextsmall}[1] {
	\begin{tabular*}{0.8\mpwidth}{p{0.8\mpwidth}}
		\parbox{0.8\mpwidth}{#1}
	\end{tabular*}
}
%----------------------------------------------------------------------------------------
%	CV SECTION
%----------------------------------------------------------------------------------------

% Renders a a CV section headline with a nice underline in main color.
% param 1: section title
\newlength{\barw}
\newcommand{\cvsection}[1] {
	\vspace{14pt}
	\settowidth{\barw}{\textbf{\LARGE #1}}
	\cvtext{
		\textbf{\LARGE{\textcolor{darkcol}{#1}}}\\[-4pt]
		\textcolor{accentcol}{ \rule{\barw}{1.5pt} } \\
	}
}

\newcommand{\cvsubsection}[1] {
	\vspace{14pt}
	\settowidth{\barw}{\textbf{\Large #1}}
	\cvtext{
		\textbf{\Large{\textcolor{darkcol}{#1}}}\\[-4pt]
		\textcolor{accentcol}{ \rule{\barw}{1.5pt} } \\
	}
}

\newcommand{\cvheadline}[1] {
	\vspace{16pt}
	\cvtext{
		\textbf{\Huge{\textcolor{accentcol}{#1}}}\\[-4pt]
		 
	}
}

\newcommand{\cvsubheadline}[1] {
	\vspace{16pt}
	\cvtext{
		\textbf{\huge{\textcolor{darkcol}{#1}}}\\[-4pt]
		 
	}
}
%----------------------------------------------------------------------------------------
%	META SKILL
%----------------------------------------------------------------------------------------

% Renders a progress-bar to indicate a certain skill in percent.
% param 1: name of the skill / tech / etc.
% param 2: level (for example in years)
% param 3: percent, values range from 0 to 1
\newcommand{\cvskill}[3] {
	\begin{tabular*}{1\mpwidth}{p{0.72\mpwidth}  r}
 		\textcolor{black}{\textbf{#1}} & \textcolor{maincol}{#2}\\
	\end{tabular*}%
	
	\hspace{4pt}
	\begin{tikzpicture}[scale=1,rounded corners=2pt,very thin]
		\fill [lightcol] (0,0) rectangle (1\mpwidth, 0.15);
		\fill [accentcol] (0,0) rectangle (#3\mpwidth, 0.15);
  	\end{tikzpicture}%
}


%----------------------------------------------------------------------------------------
%	 CV EVENT
%----------------------------------------------------------------------------------------

% Renders a table and a paragraph (cvtext) wrapped in a parbox (to ensure minimum content
% is glued together when a pagebreak appears).
% Additional Information can be passed in text or list form (or other environments).
% the work you did
% param 1: time-frame i.e. Sep 14 - Jan 15 etc.
% param 2:	 event name (job position etc.)
% param 3: Customer, Employer, Industry
% param 4: Short description
% param 5: work done (optional)
% param 6: technologies include (optional)
% param 7: achievements (optional)
\newcommand{\cvevent}[4] {
	
	% we wrap this part in a parbox, so title and description are not separated on a pagebreak
	% if you need more control on page breaks, remove the parbox
	\parbox{\mpwidth}{
		\begin{tabular*}{1\mpwidth}{p{0.66\mpwidth}  r}
	 		\textcolor{black}{\textbf{#2}} & \colorbox{accentcol}{\makebox[0.32\mpwidth]{\textcolor{white}{\textbf{#1}}}} \\
			\textcolor{maincol}{#3} & \\
		\end{tabular*}\\[8pt]
	
		\ifthenelse{\isempty{#4}}{}{
			\cvtext{#4}\\
		}
	}
	\vspace{14pt}
}

\newcommand{\cvproj}[3] {
	\parbox{\mpwidth}{
		\begin{tabular*}{1\mpwidth}{p{0.66\mpwidth}  r}
	 		\textcolor{black}{\textbf{#1}} & \\
			\textcolor{maincol}{#2} & \\
		\end{tabular*}\\[4pt]
	
		\ifthenelse{\isempty{#3}}{}{
			\cvtext{#3}\\
		}
	}
	\vspace{14pt}
}

%----------------------------------------------------------------------------------------
%	 CV META EVENT
%----------------------------------------------------------------------------------------

% Renders a CV event on the sidebar
% param 1: title
% param 2: subtitle (optional)
% param 3: customer, employer, etc,. (optional)
% param 4: info text (optional)
\newcommand{\cvmetaevent}[4] {
	\textcolor{maincol} { \cvtext{\textbf{\begin{flushleft}#1\end{flushleft}}}}

	\ifthenelse{\isempty{#2}}{}{
	\textcolor{black} {\cvtext{\textbf{#2}} }
	}

	\ifthenelse{\isempty{#3}}{}{
		\cvtext{{ \textcolor{maincol} {#3} }}\\
	}

	\cvtext{#4}\\[14pt]
}

%---------------------------------------------------------------------------------------
%	QR CODE
%----------------------------------------------------------------------------------------

% Renders a qrcode image (centered, relative to the parentwidth)
% param 1: percent width, from 0 to 1
\newcommand{\cvqrcode}[1] {
	\begin{center}
		\includegraphics[width={#1}\mpwidth]{qrcode}
	\end{center}
}


% HEADER AND FOOOTER 
%====================================
\newcommand\Header[1]{%
\begin{tikzpicture}[remember picture,overlay]
\fill[accentcol]
  (current page.north west) -- (current page.north east) --
  ([yshift=50pt]current page.north east|-current page text area.north east) --
  ([yshift=50pt,xshift=-3cm]current page.north|-current page text area.north) --
  ([yshift=10pt,xshift=-5cm]current page.north|-current page text area.north) --
  ([yshift=10pt]current page.north west|-current page text area.north west) -- cycle;
\node[font=\sffamily\bfseries\color{white},anchor=west,
  xshift=0.7cm,yshift=-0.32cm] at (current page.north west)
  {\fontsize{12}{12}\selectfont {#1}};
\end{tikzpicture}%
}

\newcommand\Footer[1]{%
\begin{tikzpicture}[remember picture,overlay]
\fill[lightcol]
  (current page.south east) -- (current page.south west) --
  ([yshift=-80pt]current page.south east|-current page text area.south east) --
  ([yshift=-80pt,xshift=-6cm]current page.south|-current page text area.south) --
  ([xshift=-2.5cm,yshift=-10pt]current page.south|-current page text area.south) --	
  ([yshift=-10pt]current page.south east|-current page text area.south east) -- cycle;
\node[yshift=0.32cm,xshift=9cm] at (current page.south) {\fontsize{10}{10}\selectfont \textbf{\thepage}};
\end{tikzpicture}%
}


%=====================================
%============================================================================%
%
%
%
%	DOCUMENT CONTENT
%
%
%
%============================================================================%
\begin{document}

\columnratio{0.31}
\setlength{\columnsep}{2.2em}
\setlength{\columnseprule}{4pt}
\colseprulecolor{white}


% LEBENSLAUF HIERE
\AtBeginShipoutFirst{\Header{CV}\Footer{1}}
\AtBeginShipout{\AtBeginShipoutAddToBox{\Header{CV}\Footer{2}}}

\newpage

\colseprulecolor{lightcol}
\columnratio{0.31}
\setlength{\columnsep}{2.2em}
\setlength{\columnseprule}{4pt}
\begin{paracol}{2}
	\begin{leftcolumn}
		%---------------------------------------------------------------------------------------
		%	META IMAGE
		%----------------------------------------------------------------------------------------
                \includegraphics[width=\linewidth]{foto\_perfil.jpeg}
        		\fcolorbox{white}{white}{\begin{minipage}[c][1.5cm][c]{1\mpwidth}
				\LARGE{\textbf{\textcolor{maincol}{\cvtext{%
Francisco \linebreak Zambrano}}}} \\[2pt]
				\normalsize{ \textcolor{maincol} {} }
			\end{minipage}}

		%---------------------------------------------------------------------------------------
		%	META SKILLS
		%----------------------------------------------------------------------------------------
				\icontext{caret-right}{12}{Providencia, Santiago, Chile}\\[6pt]
				\icontext{caret-right}{12}{Chileno-Italiano}\\[6pt]
		
		
		\cvsection{Habilidades}

						\cvskill{R}{10+ años}{0.9} \\[10pt]
								\cvskill{R \textbar{} \{tidyverse\}}{3+ años}{0.8} \\[10pt]
								\cvskill{R \textbar{} \{tidymodels\}}{1+ años}{0.7} \\[10pt]
								\cvskill{Python}{2 años}{0.3} \\[10pt]
								\cvskill{Amazon Web Service (AWS\textbar EC2)}{3 años}{0.5} \\[10pt]
								\cvskill{Dockers}{2 años}{0.3} \\[10pt]
								\cvskill{SIG}{10 años}{0.8} \\[10pt]
								\cvskill{Rmarkdown}{6 años}{0.7} \\[10pt]
								\cvskill{Quarto}{1 años}{0.6} \\[10pt]
								\cvskill{Análisis datos espaciales}{10 años}{0.8} \\[10pt]
								\cvskill{R-Shiny}{6 años}{0.5} \\[10pt]
						
		\cvsection{Software}

						\icontext{caret-right}{12}{Git}\\[6pt]
								\icontext{caret-right}{12}{RStudio - Positron}\\[6pt]
								\icontext{caret-right}{12}{VS Code}\\[6pt]
								\icontext{caret-right}{12}{Terminal}\\[6pt]
								\icontext{caret-right}{12}{QGIS}\\[6pt]
								\icontext{caret-right}{12}{SAGA}\\[6pt]
								\icontext{caret-right}{12}{SNAP}\\[6pt]
						
		\cvsection{Manejo de datos satelitales}

						\icontext{caret-right}{12}{MODIS}\\[6pt]
								\icontext{caret-right}{12}{ERA5/ERA5-Land}\\[6pt]
								\icontext{caret-right}{12}{CHIRPS}\\[6pt]
								\icontext{caret-right}{12}{Sentinel-1/2}\\[6pt]
								\icontext{caret-right}{12}{Landsat 7/8/9}\\[6pt]
								\icontext{caret-right}{12}{SoilGrid}\\[6pt]
								\icontext{caret-right}{12}{CMIP6}\\[6pt]
						
		\cvsection{Curso Especialización en Ciencia de Datos (Coursera)}

						\icontext{caret-right}{12}{R-programming}\\[6pt]
								\icontext{caret-right}{12}{Getting and cleaning data}\\[6pt]
								\icontext{caret-right}{12}{Exploratory data analysis}\\[6pt]
								\icontext{caret-right}{12}{Reproducible Research}\\[6pt]
								\icontext{caret-right}{12}{Statistical Inference}\\[6pt]
								\icontext{caret-right}{12}{Regression Models}\\[6pt]
								\icontext{caret-right}{12}{Practical Machine Learning}\\[6pt]
								\icontext{caret-right}{12}{Developing Data Products}\\[6pt]
						
		\cvsection{Premios}

						\icontext{caret-right}{12}{Hackaton Winner in the OpenGeoHub Summer
School, Siegburg, Germany, 2022.}\\[6pt]
								\icontext{caret-right}{12}{Beca Doctorado, Agencia Nacional de
Investigación y Desarrollo, Chile, 2014.}\\[6pt]
						
		\cvsection{Educación}

						\cvmetaevent{03/2014 - 09/2017}{Dr.Ingenieria Agrícola mención Recursos
Hídricos}{Universidad de Concepción}{Tesis: Sequía Agrícola en Chile.
Desde la evaluación hacia la predicción usando datos satelitales}
								\cvmetaevent{03/2000 - 09/2007}{Ingenieria Civil}{Universidad de
Concepción}{}
						
		\cvsection{Contacto}

						\icontext{map-marker}{16}{Providencia, Santiago, Chile}\\[6pt]
								\icontext{phone}{16}{+56 9684 77864}\\[6pt]
								\icontext{envelope}{16}{frzambra@gmail.com}\\[6pt]
								\iconhref{home}{16}{francisco-zambrano.cl}{https://francisco-zambrano.cl}\\[6pt]
								\iconhref{github}{16}{frzambra}{https://github.com/frzambra}\\[6pt]
						
	\end{leftcolumn}

	\begin{rightcolumn}
		\cvsection{Resumen} \vspace{4pt}

\cvtext{%
Científico de datos con más de diez años de experiencia en el manejo y
análisis de grandes volúmenes de datos. Especialista en machine
learning, deep learning y análisis geoespacial aplicado a la toma de
decisiones. Experiencia comprobada en desarrollo de modelos predictivos,
visualización de datos y gestión de proyectos tecnológicos de alto
impacto. He liderado proyectos de gran escala y publicado en revistas
científicas de alto impacto, lo que respalda mi capacidad para generar
conocimiento aplicado y resultados accionables.}

\cvsection{Experiencia en Ciencia de Datos} \vspace{4pt}

\cvevent{2025}{PM 2.5}{https://frzambra.shinyapps.io/app_pm25 \newline  Developer}{I modeled monthly particulate matter 2.5 for continental Chile using Machine Learning algorithms, data from the SINCA network (National Air Quality Information System), and satellite data. In addition, I developed a prototype of a web platform for visualizing the results.}

\cvevent{2025}{SatOri}{https://s4tori.cl/app \newline  Director}{I led and worked on spatial data processing for the development of the Satellite System for Irrigation Optimization in fruit orchards (SatOri). I modeled the spatial and temporal variation of xylem water potential in cherry trees using Sentinel-2 satellite data and Machine Learning algorithms.}

\cvevent{2022}{ODES-Chile}{https://odes-chile.org/app/unidades \newline  Director}{I led and worked on spatial data processing for the drought observatory for agriculture and biodiversity in Chile (ODES). I processed ERA5-Land climate data and calculated drought indicators for all continental Chile. I generated spatial aggregation across different hydrological and administrative units.}

\cvevent{2021}{IAF app}{https://frzambra.shinyapps.io/iaf_shinyapp/ \newline  Developer}{I developed a mobile-adapted application that allows calculating the area of a leaf, supporting the calculation of the Leaf Area Index (LAI).}

\cvevent{2017}{Drought prediction}{https://frzambra.shinyapps.io/drought_prediction/ \newline  Developer}{I developed a web application in which I implemented the results of my research article on the prediction of agricultural drought in Chile.}

\newpage

\cvsection{Otras experiencias profesionales} \vspace{4pt}

\cvevent{02/2018 - 08/2025}{Académico Asociado}{Centro de Observación de la Tierra \newline Hemera - Universidad Mayor}{Me adjudiqué y dirigí proyectos financiados por ANID por mas de 600 millones, entre ellos un Fondecyt de Iniciación, un FONDEF IDeA y un fondo de sequía. Coordiné el desarrollo de las plataformas ODES-Chile y SatOri, enfocadas en la adaptación al cambio climático mediante observación de la tierra y análisis espacial. En el ámbito académico, impartí cursos de pregrado y postgrado en SIG (QGIS) y ciencia de datos espaciales con R, formando estudiantes en tecnologías aplicadas a la gestión ambiental y territorial.}

\cvevent{09/2016 - 12/2016}{Investigador Doctoral Visitante}{Facultad de Ciencias de Geoinformación y Observación de la Tierra (ITC) \newline  Universidad de Twente, Enschede, The Netherlands}{Lideré un estudio para predecir la disminución de la productividad agrícola inducida por sequías en Chile, integrando series temporales de datos satelitales (MODIS, CHIRPS) y técnicas avanzadas de análisis espacial. Los resultados de esta investigación fueron publicados en el journal Remote Sensing of Environment.}

\cvevent{01/2016 - 06/2016}{Investigador Doctoral Visitante}{Centro de Tecnologías Avanzadas de Información para la Gestión de Tierras (CALMIT) \newline  Centro Nacional de Mitigación de Sequía (NDMC) \newline  Universidad de Nebraska, Lincoln, Nebraska, Estados Unidos}{Lideré un estudio sobre la evaluación de productos satelitales para estimar la precipitación en Chile y su aplicabilidad en el monitoreo de sequías. Los resultados fueron publicados en el journal Atmospheric Research.}

\cvevent{14/2012 - 03/2015}{Investigador Asistente}{Centro Regional de Investigación Quilamapu  \newline Instituto Nacional de Investigaciones Agropecuarias (INIA)}{Procesé y analicé datos de estaciones climáticas y satelitales para el estudio y monitoreo de la sequía en Chile. Además, automatice la generación de reportes mensuales sobre sequía y agroclima, incorporados en los informes agroclimáticos regionales.}

\cvevent{09/2007 - 12/2012}{Varios}{Servicios públicos \newline  CNR | DGA | INDAP}{He trabajado como ingeniero en servicios públicos como la Dirección General de Aguas (DGA), Comisión Nacional de Riego (CNR) e Instituto de Desarrollo Agropecuario (INDAP), en diferentes regiones de Chile, en temas relacionados con recursos hídricos, agricultura y organizaciones de usuarios de agua (OUAs).}

\cvsection{Proyectos concursables adjudicados} \vspace{4pt}

\cvtext{%
Proyectos adjudicados por la Agencia Nacional de Investigación y
Desarrollo (ANID)}

\cvevent{01/2025 -10/2025}{Crea Ciencia 2030}{Director}{Título: Impacto del cambio climático en fenología de paltos y el bosque nativo esclerófilo según acceso a agua subterránea potencial en la cuenca del río Aconcagua}

\cvevent{03/2022 - 10/2023}{ODES-Chile (FSEQ210022)}{Director}{Creamos ODES-Chile un observatorio de sequía multiescalar para Chile, un sistema de alerta temprana para mitigar impactos agrícolas y ecológicos (https://odes-chile.org).}

\cvevent{03/2022 - 12/2024}{SatOri (ID21I10297)}{Director}{Creamos SatOri un sistema sateltal para la optimización de riego en cerezos (https://s4tori.cl).}

\cvevent{03/2020 - 03/2022}{Fondecyt Iniciación 11190360}{Investigador principal}{Dirijí investigación en la que se evaluó la predicción de biomasa en trigo y maíz mediante el uso de datos satelitales y técnicas de machine learning}

\cvevent{03/2023 - 03/2025}{Fondecyt Postdoctorado}{Investigador patrocinante}{Patrociné el proyecto titulado `Evaluación de la disponibilidad hídrica actual y futura para la agricultura y los ecosistemas terrestres bajo diferentes escenarios de uso del suelo en la cuenca del aconcagua: hacia la adaptación a la sequía.`}

\cvevent{03/2021 - 12/2024}{Fondecyt Regular (1210526)}{Co-investigador}{Título: Sistema multivariado de monitoreo de sequía: modelización biofísica, teledetección e información hidroclimática para el análisis y predicción de sequías en agricultura.}
			\end{rightcolumn}
\end{paracol}



\end{document}
